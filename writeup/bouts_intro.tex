\documentclass[11pt]{article}\usepackage[]{graphicx}\usepackage[]{color}
%% maxwidth is the original width if it is less than linewidth
%% otherwise use linewidth (to make sure the graphics do not exceed the margin)
\makeatletter
\def\maxwidth{ %
  \ifdim\Gin@nat@width>\linewidth
    \linewidth
  \else
    \Gin@nat@width
  \fi
}
\makeatother

\definecolor{fgcolor}{rgb}{0.345, 0.345, 0.345}
\newcommand{\hlnum}[1]{\textcolor[rgb]{0.686,0.059,0.569}{#1}}%
\newcommand{\hlstr}[1]{\textcolor[rgb]{0.192,0.494,0.8}{#1}}%
\newcommand{\hlcom}[1]{\textcolor[rgb]{0.678,0.584,0.686}{\textit{#1}}}%
\newcommand{\hlopt}[1]{\textcolor[rgb]{0,0,0}{#1}}%
\newcommand{\hlstd}[1]{\textcolor[rgb]{0.345,0.345,0.345}{#1}}%
\newcommand{\hlkwa}[1]{\textcolor[rgb]{0.161,0.373,0.58}{\textbf{#1}}}%
\newcommand{\hlkwb}[1]{\textcolor[rgb]{0.69,0.353,0.396}{#1}}%
\newcommand{\hlkwc}[1]{\textcolor[rgb]{0.333,0.667,0.333}{#1}}%
\newcommand{\hlkwd}[1]{\textcolor[rgb]{0.737,0.353,0.396}{\textbf{#1}}}%
\let\hlipl\hlkwb

\usepackage{framed}
\makeatletter
\newenvironment{kframe}{%
 \def\at@end@of@kframe{}%
 \ifinner\ifhmode%
  \def\at@end@of@kframe{\end{minipage}}%
  \begin{minipage}{\columnwidth}%
 \fi\fi%
 \def\FrameCommand##1{\hskip\@totalleftmargin \hskip-\fboxsep
 \colorbox{shadecolor}{##1}\hskip-\fboxsep
     % There is no \\@totalrightmargin, so:
     \hskip-\linewidth \hskip-\@totalleftmargin \hskip\columnwidth}%
 \MakeFramed {\advance\hsize-\width
   \@totalleftmargin\z@ \linewidth\hsize
   \@setminipage}}%
 {\par\unskip\endMakeFramed%
 \at@end@of@kframe}
\makeatother

\definecolor{shadecolor}{rgb}{.97, .97, .97}
\definecolor{messagecolor}{rgb}{0, 0, 0}
\definecolor{warningcolor}{rgb}{1, 0, 1}
\definecolor{errorcolor}{rgb}{1, 0, 0}
\newenvironment{knitrout}{}{} % an empty environment to be redefined in TeX

\usepackage{alltt}

\usepackage{hyperref, lastpage, fancyhdr,multicol,caption,subcaption,tabularx}
\usepackage{amsmath,graphicx}
\usepackage{float}

\usepackage{geometry}
\usepackage{pdflscape}



\topmargin      -1.5cm   % read Lamport p.163
\oddsidemargin  -0.04cm  % read Lamport p.163
\evensidemargin -0.04cm  % same as oddsidemargin but for left-hand pages
\textwidth      16.59cm
\textheight     23.94cm
\parskip         7.2pt   % sets spacing between paragraphs
\parindent         0pt   % sets leading space for paragraphs
\pagestyle{empty}        % Uncomment if don't want page numbers
\pagestyle{fancyplain}

\usepackage{natbib} %need this for bibtex
\IfFileExists{upquote.sty}{\usepackage{upquote}}{}

\lhead{}
\chead{}
\rhead{}

\usepackage{setspace} %for double spacing
\doublespacing
\IfFileExists{upquote.sty}{\usepackage{upquote}}{}
\begin{document}

\title{Bayesian Semi-Parametric Energy Balance Measurement Error Models}
\author{Daniel Ries}
\date{\today}
\maketitle

Our understanding of the importance of physical activity has exploded in recent years. This is not suprising given that according to the Center for Disease Control (CDC) over 70\% of Americans age 20 and over are overweight or obese, over 37\% of those being obese alone \cite{obese}. To make matters worse, over 20\% of teenagers are obese and 9\% of children between 2 and 5 years old are already obese \cite{obese}.  Physical activity (or lack of) for Americans has become such an important issue that in 2008, the Department of Health and Human Services issued the \emph{2008 Physical Activity Guidelines for Americans} which can be found at \url{https://health.gov/paguidelines/guidelines/}. This was the first time that guidlines were set by the United States government. By comparison, the first nutritional guidelines were set in 1977 with the \emph{Dietary Goals for the United States}, otherwise known as the McGovern Report \cite{nutrition}. The report advises adults to do at least 150 minutes each week of moderate-intensity activity or 75 minutes of vigorous-intensity activity or some combination of the two. Furthermore, they recommend these activities come in spurts of at least 10 minutes, or what we refer to as \emph{bouts}. They define moderate-intensity to be a 5 or 6 on a scale of 0 to 10; they use brisk walking as an example. Vigorous-intensity is a 7 or 8 on the same scale, and they use jogging or lap swimming as examples. More precise definitions are given, and we will define and use these later in our analysis. In addition, it recommends doing muscle-strengthening activities that involve all major muscle groups twice or more per week. These are the minimum guidelines, and the report advises that the more the better. Policy makers are beginning to understand the role physical activity (like proper nutrition) has on treating diseases as well as preventive care. Regular physical activity has been linked to prevention and treatment of cardiovascular disease\cite{odphp}\cite{warburton}\cite{reiner}, diabetes \cite{odphp}\cite{warburton}\cite{reiner}, cancer \cite{odphp}\cite{warburton}, hypertension \cite{warburton}, osteoporosis \cite{odphp}\cite{warburton}, depression \cite{odphp}\cite{warburton}, obesity \cite{warburton}\cite{reiner}, and Alzheimer's \cite{reiner}.

A topic that becomes of interest is assessing how many Americans are adheering to these guidelines? And are there certain subpopulations that are more or less likely to adhere to the guidelines? These are important questions for policy makers because even if the guidelines are there, it doesn't mean anyone is paying attention. If physicaly activity has all these nice benefits, then it is in policy makers interest to ensure Americans are complying, and if there's certain groups that are systematically not adhereing, it would be beneficial for policy makers to know in order to target certain subpopulations.  In order to answer these questions, data on individuals' physical activity has to be gathered along with demographic information. Although gold standard measurements exist for measuring one's energy expenditure (EE), it is very costly and studies typically use either self-reports or some kind of wearable device or both. All of these options have measurement error involved which is known to affect results and inferences. Therefore one cannot simply use these measurements and answer the above questions without doing appropriate statistical adjustments. 


talk about bouts

talk about how this is much more common in nutrition

We propose a measurement error model to estimate the distribution of moderate to vigorous physical activity in adults while accounting for demographic information. This will allow us to estimate the compliance with the guidelines and understand which groups are more likely to comply with the guidelines. In addition, we will develop calibration methods so future studies will not need to use expensive measures of EE, or only a small subsample with the more expensive measure. 


\clearpage
\bibliographystyle{unsrt} %style of bibliography
\bibliography{bouts} %tells BibTeX to create a list of references using

\end{document}
